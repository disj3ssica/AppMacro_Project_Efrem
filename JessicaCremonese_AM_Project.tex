\documentclass[10pt,a4paper,draft]{article}
\usepackage{xcolor}
\usepackage{amsmath}
\usepackage{csquotes}
\raggedbottom % Remove extra vertical space between sections

\begin{document}

% Title Page
\begin{titlepage}
    \centering
    \vspace*{2cm}
    {\LARGE Example Title \par}
    \vspace{2cm}
    {\large Jessica Cremonese \par}
    \vspace{1cm}
    {\large Università degli Studi di Padova \par}
    \vspace{1cm}
    {\large Applied Macroeconomics (Mod. B) - 2022-2023 \par}
    \vspace{1cm}
    {\large Date: \par}
    {\large \today \par}
\end{titlepage}


\newpage
\section{Introductory remarks}

% WHAT IS A PRICE PUZZLE?               [DONE]
In the study of monetary policy (MP) shocks, a "price puzzle" refers to a situation where the response of prices to a MP shock is inconsistent with the predictions of standard economic models. Specifically, it refers to an unexpected short-term relationship between MP actions, such as changes in interest rates, and the behavior of prices. A contractionary MP shock, such as an increase in interest rates, is expected to dampen aggregate demand and result in lower output and inflation. However, researchers have found instances where the response of prices to MP shocks is contrary to these expectations. For example, after a contractionary MP shock, prices may increase in the short term instead of decreasing as predicted. This deviation from the predicted relationship between MP and prices is referred to as a price puzzle. Some explanations for the price puzzle could be imperfect competition, nominal rigidities, delayed price adjustment, supply-side effects, and other factors that can lead to short-term departures from standard economic models. Understanding the causes and implications of the price puzzle is crucial for improving our understanding of the dynamics of MP and its impact on the economy.


% Give intuition about why MP shocks affect the GFC: INTERNATIONAL SHOCK TRANSMISSION.
The global financial cycle (GFC) responds to US MP. 



% DEFINING INFORMATION SHOCKS


% ON MAREY 2020
    %   DOES MAREY FIND PRICE OR OUTPUT PUZZLES? CHECK AND COMMENT.

Miranda-Agrippino and Rey (2020) explores the international transmission of MP through global intermediaries and global asset prices, finding that there are powerful financial spillovers from US MP to other countries. After a US MP shock the GFC responds with significant fluctuations in financial activities, global banks deleveraging both domestically and abroad, increased aggregate risk aversion and a contraction of international credit. The authors document the existence of a common factor in the GFC that accounts for 20\% of the common variation in the price of risky assets worldwide.

Miranda-Agrippino and Rey (2020) identify MP shocks using market price revisions around the Federal Open Market Committee (FOMC) announcements, stating that, in a market with informational asymmetries, these surprises represent a reaction to the central bank (CB) conveying implicit information about market fundamentals.

    

% ON MARICCO 2021
Miranda-Agrippino and Ricco (2021) contribute to the literature by providing an empirical specification to account for informational asymmetries that generate an \enquote{information channel} for MP choices. The authors state that an emerging feature of models with imperfect information is that high frequency instruments are predictable and autocorrelated, due to the sluggish adjustment of expectations. 

Therefore, they define MP shocks as the exogenous shifts in the policy instrument that are unforecastable and independent from the CB's information set about economic conditions. The instrument for MP shocks is constructed by projecting monetary surprises on their own lags and on the CB's information set. 

With this identification strategy they find no output nor price puzzles. Furthermore, they find that a monetary contracion is unequivocally and significantly recessionary, with output contracting immediately and significantly, as well as prices, domestic demand, labor market conditions, investments and household wealth. There is evidence of a credit channel which magnifies these effects through credit and financial markets. 



% MAREY 2020 in light of MARICCO 2021:



\section{Application}
s


\section{Discussion}
% In light of MARICCO, how should we think about the MAREY paper? Answer this to provide motivazion with what you do with the data.


s




















\end{document}