\documentclass[10pt,a4paper,draft]{article}
\usepackage{xcolor}
\usepackage{amsmath}
\raggedbottom % Remove extra vertical space between sections

\begin{document}

% Title Page
\begin{titlepage}
    \centering
    \vspace*{2cm}
    {\LARGE Example Title \par}
    \vspace{2cm}
    {\large Jessica Cremonese \par}
    \vspace{1cm}
    {\large Università degli Studi di Padova \par}
    \vspace{1cm}
    {\large Applied Macroeconomics (Mod. B) - 2022-2023 \par}
    \vspace{1cm}
    {\large Date: \par}
    {\large \today \par}
\end{titlepage}


\newpage
\section{Intro}

% WHAT IS A PRICE PUZZLE? 
    % begin with a discussion of the price puzzle
        % In the study of monetary policy shocks, a "price puzzle" refers to a situation where the response of prices to a monetary policy shock is inconsistent with the predictions of standard economic models. Specifically, it refers to an unexpected short-term relationship between monetary policy actions, such as changes in interest rates, and the behavior of prices.a contractionary monetary policy shock, such as an increase in interest rates, is expected to dampen aggregate demand and result in lower output and inflation. However, in some empirical studies, researchers have found instances where the response of prices to monetary policy shocks is contrary to these expectations. For example, after a contractionary monetary policy shock, prices may increase in the short term instead of decreasing as predicted. This deviation from the predicted relationship between monetary policy and prices is referred to as a price puzzle. Researchers have explored various explanations for the price puzzle, including imperfect competition, nominal rigidities, delayed price adjustment, supply-side effects, and other factors that can lead to short-term departures from standard economic models. Understanding the causes and implications of the price puzzle is crucial for improving our understanding of the dynamics of monetary policy and its impact on the economy.



%   ON MAREY 2020
Miranda-Agrippino and Rey (2020) explores the international transmission of monetary policy through global intermediaries and global asset prices, finding that there are powerful financial spillovers from US monetary policy to other countries'. In response to a US monetary policy (MP) shock they observe effects on the global financial cycle (GFC) as shown by significant fluctuations in financial activities, global banks deleveraging both domestically and abroad, increased aggregate risk aversion and a contraction of international credit. The authors' document the existence of a common factor in the GFC that accounts for 20\% of the common variation in the price of risky assets worldwide.

Miranda-Agrippino and Rey (2020) identify MP shocks using market price revisions around the Federal Open Market Committee (FOMC) announcements, stating that, in a market with informational asymmetries, these surprises represent a reaction to the central bank conveying implicit information about market fundamentals.

    % note: MP *shock* defined in MAREY as the high frequency market revisions around FOMC (federal open market committee) announcements. basically: check market prices 30m before announcement, then check market prices after announcement; the difference is the shock.

%   ON MARICCO 2021
Miranda-Agrippino and Ricco (2021) provides an empirical specification to account for information asymmetries that generate an "information channel" for MP. The authors state that an emerging feature of models with imperfect information is that high frequency instruments are predictable and autocorrelated, due to the sluggish adjustment of expectations. 
%MP SHOCK:
Therefore, they define MP shocks as the exogenous shifts in the policy instrument that are unforecastable and independent from the CB's information set about economic conditions. The instrument for MP shocks is constructed by projecting monetary surprises on their own lags and on the CB's information set. 

With this identification strategy they find no output nor price puzzles. Furthermore, they find that a monetary contracion is unequivocally and significantly recessionary, with output contracting immediately and significantly, as well as prices, domestic demand, labor market conditions, investments and household wealth. There is evidence of a credit channel which magnifies these effects through credit and financial markets. 

    % pp76 <<high frequency market based surprises around policy announcements correlate with CB's private macro forecasts. [...] evidence of the signaling channel>> i.e. the indirect transfer of info from CB to private agents





\section{Application}
s


\section{Discussion}
s




















\end{document}