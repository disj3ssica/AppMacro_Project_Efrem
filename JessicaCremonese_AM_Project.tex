\documentclass[10pt,a4paper,draft]{article}
\usepackage{xcolor}
\usepackage{amsmath}
\usepackage{csquotes}
\raggedbottom % Remove extra vertical space between sections

\begin{document}

% Title Page
\begin{titlepage}
    \centering
    \vspace*{2cm}
    {\LARGE Information Shocks and the Global Financial Cycle \par}
    \vspace{2cm}
    {\large Jessica Cremonese \par}
    \vspace{1cm}
    {\large Università degli Studi di Padova \par}
    \vspace{1cm}
    {\large Applied Macroeconomics (Mod. B) - 2022-2023 \par}
    \vspace{1cm}
    {\large Date: \par}
    {\large \today \par}
\end{titlepage}


\newpage
\section{Introductory remarks}
Unprecedented financial globalization and the emergence of global intermediaries have affected her national financial markets work and the transmission of monetary policy shocks between interconnected countries. The United states dollar represents the dominant currency on the international scenario, so much so that US monetary policy has been documented to have spillover effects to other countries. Therefore, monetary policy decisions in the US directly influence the global financial cycle. Domestic decisions affect major global banks' portfolios through assets being largely denominated in dollars. Consequently, lending and credit decisions are affected by monetary policy decisions, corporate spreads and prices are affected as well as agents' risk propensity. 

% WHAT IS A PRICE PUZZLE?
In the study of monetary policy (MP) shocks, a \enquote{price puzzle} refers to a situation where the response of prices to a MP shock is inconsistent with the predictions of standard economic models. Specifically, it refers to an unexpected short-term relationship between MP actions, such as changes in interest rates, and the behavior of prices. A contractionary MP shock, such as an increase in interest rates, is expected to dampen aggregate demand and result in lower output and inflation. However, researchers have found instances where the response of prices to MP shocks is contrary to these expectations. For instance, after a contractionary MP shock, prices may increase in the short term instead of decreasing as predicted. This deviation from the predicted relationship between MP and prices is referred to as a price puzzle. Understanding the causes and implications of the price puzzle is crucial for improving our understanding of the dynamics of MP and its impact on the economy.

% MAREY 2020
Miranda-Agrippino and Rey (2020) investigates the spillovers of us monetary policy. MP shocks are identified using an external instrument constructed from high-frequency asset prices adjustments around FOMC (Federal Open Market Committee) announcements. 
The instrument is valid only under the assumption that market agents can correctly discern monetary policy decisions from policy actions. If informational asymmetries are present, the high-frequency surprises are also a funcion of the \enquote{FED information effect}, i.e. information shocks about the economic fundamentals that are implicitly conveyed by the central bank's announcements. Failure to account for information asymmetries results in price and output puzzles. The authors opt for analysing the transmission of MP shocks via a rich information VAR, which reduces issues such as omitted variables and claims to manage imperfect information. 

% possibly report results of MAREY (2020)
Their results on domestic resposes report that a contractionary shosk depresses prices and output, as well as housing investments. Unemployment rises and prices adjust downward. The price puzzle is less pronounced with the authors' approach with respect to standard IV approaches.
The effect on the global financial cycle are shown through the responses of global risky asset prices, which contract via the common global factor, and the increase in aggregate risk aversion. After a US MP contraction there is a sharp decrease in credit provision and global capital inflows. Global intermediaries react faster than domestically oriented retail banks, however both show a contraction in leverage. 


% MARICCO 2021
    
In the presence of incorrect accounting for imperfect information, the estimated dynamic responses to shocks are dependent on the choice of instrument, sample and empirical specification of choice. Miranda-Agrippino and Ricco (2021) propose an identification strategy that is robust to the presence of imperfect information between the public and the central bank, which generates an information channel for monetary policy actions. 
As evidence of the information channel of monetary policy, the authors find that high-frequency market based surprises around policy announcements tend to correlate with the central bank's private macroeconomic forecasts. Consequently, they define monetary policy shocks as the exogenous shifts in the policy instrument that surprise market participants and are unforecastable, as well as being uncorrelated with the central banks systematic response to the macroeconomic outlook.  This way they are able to untangle monetary policy actions from information shocks. 
Their results show behavior of dynamic responses coherent with economic theory and no evidence of output or price puzzles. Furthermore, they find that a monetary contracion is unequivocally and significantly recessionary, with output contracting immediately and significantly, as well as prices, domestic demand, labor market conditions, investments and household wealth. There is evidence of a credit channel which magnifies these effects through credit and financial markets, the yield curve flattens, borrowing costs and corporate spreads rise.





% MAREY 2020 in light of MARICCO 2021:
% ******************   CHECK VERY WELL WITH DARIO   *****************
Miranda-Agrippino and Ricco (2021) sheds light on the dynamics of Miranda-Agrippino and Rey (2020). Specifically, the average monthly surprise instrument employed in the latter still shows a slight price puzzle which entirely disappears in the informationally robust identification of the 2021 paper, where a MP shock immediately contracts industrial production and raises the unemployment rate.



% Final intro paragraph
In the following section I will estimate a VAR model to analyze the effects of an information shock as defined by Miranda-Agrippino and Ricco (2021).

\section{Application}
s


\section{Discussion}
% In light of MARICCO, how should we think about the MAREY paper? Answer this to provide motivazion with what you do with the data.


s


















\section{References}
\begin{itemize}
    \item Miranda-Agrippino, Silvia, and Hélene Rey. \enquote{US monetary policy and the global financial cycle.} The Review of Economic Studies 87.6 (2020): 2754-2776.
    \item Miranda-Agrippino, Silvia, and Giovanni Ricco. \enquote{The transmission of monetary policy shocks.} American Economic Journal: Macroeconomics 13.3 (2021): 74-107.
\end{itemize}

\end{document}