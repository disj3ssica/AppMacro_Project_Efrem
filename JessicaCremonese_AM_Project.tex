\documentclass[11pt,a4paper]{article}
% Doc settings
    \usepackage[margin=2.3cm]{geometry}
    \usepackage{xcolor}
    \usepackage{amsmath}
    \usepackage{csquotes}
    \usepackage{setspace}
    \usepackage{url}
    \usepackage{graphicx}
    \usepackage{geometry}
    \usepackage{subcaption}

    \setstretch{1.25}    % Set line spacing to 1.5 times the font size
    \raggedbottom       % Remove extra vertical space between sections

\begin{document}

% Title Page
\begin{titlepage}
    \centering
    \vspace*{2cm}
    {\LARGE Information Shocks and the Global Financial Cycle \par}
    \vspace{2cm}
    {\large Jessica Cremonese \par}
    \vspace{1cm}
    {\large Università degli Studi di Padova \par}
    \vspace{1cm}
    {\large Applied Macroeconomics (Mod. B) - 2022-2023 \par}
    \vspace{1cm}
    {\large Date: \par}
    {\large \today \par}
\end{titlepage}


\newpage
\section{Introductory remarks}    
Unprecedented financial globalization, coupled with the emergence of global intermediaries, has profoundly influenced the functioning of national financial markets and the transmission mechanisms of policy shocks among increasingly interconnected nations.
The United States dollar has traditionally held a position of unrivaled dominance in the international arena, to such an extent that US monetary policy has been observed to have spillover effects on other economies.
Due to the extensive dollar-denominated assets, domestic decisions significantly impact the portfolios of major global banks, thereby affecting lending and credit choices, corporate spreads, price levels, and agents' risk propensity.
For this reason, macroeconomists are increasingly interested in understanding the movements of capital flows and the transmission of policy across borders. 

% WHAT IS A PRICE PUZZLE?
In the study of monetary policy (MP) shocks, a \enquote{price puzzle} refers to a situation where the response of prices to a MP shock is inconsistent with the predictions of standard economic theory. Specifically, it refers to an unexpected short-term relationship between MP actions and the behavior of prices and output. 
A contractionary MP shock, such as an increase in interest rates, is expected to dampen aggregate demand and result in lower output and inflation. However, researchers have found instances where the response of prices to MP shocks is contrary to these expectations. 
Understanding the causes and implications of price puzzles is crucial for improving our understanding of the dynamics of MP and its impact on the economy.

% MAREY 2020
A recent paper by Miranda-Agrippino and Rey (2020) the interplay of US monetary policy and the existence of a Global Financial Cycle (GFC). 
The authors analyse the transmission of MP shocks via a rich information Bayesian VAR with an external instrument, which claims to reduce issues of omitted variables and manage imperfect information. 
        % MA IN REALTA' NON RIESCONO PERCHE' SI ASSUME CHE CB HA PERFECT INFORMATION, MA NON E' COSI'
MP shocks are identified from high-frequency asset prices adjustments around FOMC (Federal Open Market Committee) announcements. 
Such an instrument is valid only under the assumption that market agents can correctly discern monetary policy decisions from policy actions. If informational asymmetries are present, the high-frequency surprises are also a funcion of the \enquote{FED information effect}: information shocks about the economic fundamentals that are implicitly conveyed by the central bank's announcements. 
Thus, the authors are using central bank's (CB) announcements to isolate the MP shocks. However, they assume that announcements are also informative with respect to the CB's internal macroeconomic assessment. 


% results of MAREY (2020)
The findings show the existence of a common global factor which accounts for roughly 20\% of the variability in prices of risky assets. 
In terms of domestic responses, they report that a contractionary shock depresses prices and output, and housing investments, accompanied by an increase in unemployment.
significantly, their work highlights a distinct reaction in global prices of risky assets, indicating the existence of spillover effects.
Specifically, following a contraction in US monetary policy, there is an upsurge in risk aversion, a substantial reduction in credit provision, and a decrease in global capital inflows. Global intermediaries react faster than domestically oriented retail banks, however contract leverage.


% MARICCO 2021    
In the presence of incorrect accounting for imperfect information, the estimated dynamic responses to shocks are dependent on the choice of instrument, sample and empirical specification of choice. Miranda-Agrippino and Ricco (2021) propose an identification strategy that is robust to the presence of imperfect information between the public and the CB.
High-frequency market based surprises around policy announcements tend to correlate with the CB's private macroeconomic forecasts. Consequently, they define the unexpected component of monetary policy shocks as the exogenous shifts in the policy instrument that surprise market participants and are unforecastable, as well as being uncorrelated with the CB's systematic response to the macroeconomic outlook.  This allows to disentangle monetary policy actions from information shocks. 
Their results show behavior of dynamic responses coherent with economic theory and no evidence of output or price puzzles. Furthermore, they find that a monetary contracion is unequivocally and significantly recessionary, with output contracting immediately and significantly, as well as prices, domestic demand, labor market conditions, investments and household wealth. There is evidence of a credit channel which magnifies these effects through credit provision and financial markets, the yield curve flattens, borrowing costs and corporate spreads rise.

% secondo Dario:
% limite di REY: external instrument IV + 
    %%          NON TIENE CONTO DEL SIGNALING EFFECT DELLA CB !!!
% RICCO usa proxy, quindi proxy entra come variabile

In order to estimate the effect of an information shock on the GFC and a number of relevant variables, I will exploit the information shock proxy INFO\_FF4 derived by Miranda-Agrippino and Ricco (2021).
In section 2 I will estimate a VAR model to obtain the relevant impulse response functions (IRFs) to the shock.
Section 3 will provide a discussion of results and conclude.



\section{Identification and estimation}
% Technical details
In order to estimate the effects of an information shock on the GFC I employ a VAR model with three lags, a constant and a trend. The model is estimated using Ambrogio Cesa-Bianchi's VAR Toolbox for MatLab. 

%Data
The dataset covers monthly observations from 1991 to 2015. I include all the available variables in the model: the unemployment rate, a logarithmic transformation of CPI, the 3-month policy rate and the GFC, 10-year-3-month spread.

Time series and IRFs for an analogous VAR model based on the monetary policy proxy, MPI\_FF4, have already been shown in class. For this reason, I do not report my output for IRFs in the case of a MP shock and I will refer to the provided material for any comparison.

% insert vectorial figures

\newpage %IRFs

\begin{figure}
    \centering
    \includegraphics[scale=.5]{Graphs/IRF68.jpeg}
    \caption{IRFs for an information shock - 68\% confidence bands}
    \label{fig:IRF68}
  \end{figure}

  \newpage %ADDITIONAL

  \begin{figure}[ht]
    \centering

    \begin{subfigure}{\textwidth}
        \includegraphics[scale=.42]{Graphs/forecast_err_variance_decomp.jpg}
        \caption{Forecast Error Variance Decomposition}
        \label{fig:fevd}
    \end{subfigure}
    
    \vspace{0.5cm} % Adjust the vertical space between the rows
    
    \begin{subfigure}{\textwidth}
        \includegraphics[scale=.42]{Graphs/historical_decomp_nolegend.jpg}
        \caption{Historical Decomposition of Information Shock}
        \label{fig:hd}
    \end{subfigure}

    \caption{Additional Elements}
    \label{fig:additional_elements}
  \end{figure}
  

% 1. IMPOSTAZIONI DEL VAR
% 2. GRAFICO SERIE VARIABILI (SE C'E' SPAZIO)
% 



\section{Discussion}

% > DISCUSSION OF IRFS (tutte VELOCEMENTE)
% > DISCUSSIONE DELLA IRF DEL GFC (NEL DETTAGLIO)
    % > MECCANISMI POSSIBILI
    % > CONFRONTO CON LA IRF A MP SHOCK






\section*{References}
\begin{itemize}
    \item Cesa-Bianchi, Ambrogio. \url{https://github.com/ambropo/VAR-Toolbox}, VAR toolbox for MatLab
    \item Christiano, Lawrence J., Martin Eichenbaum, and Charles L. Evans. \enquote{Nominal rigidities and the dynamic effects of a shock to monetary policy.} Journal of political Economy 113.1 (2005): 1-45.
    \item Miranda-Agrippino, Silvia, and Giovanni Ricco. \enquote{The transmission of monetary policy shocks.} American Economic Journal: Macroeconomics 13.3 (2021): 74-107.
    \item Miranda-Agrippino, Silvia, and Hélene Rey. \enquote{US monetary policy and the global financial cycle.} The Review of Economic Studies 87.6 (2020): 2754-2776.
    
\end{itemize}

\end{document}